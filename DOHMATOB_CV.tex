\documentclass[11pt,a4paper]{moderncv}
\moderncvtheme[blue]{classic}                
\usepackage[utf8]{inputenc}
\usepackage[top=1.1cm, bottom=1.1cm, left=2cm, right=2cm]{geometry}
\usepackage{hyperref}
%\usepackage[scale=0.8]{geometry}
\setlength{\hintscolumnwidth}{3cm}
\setlength{\makecvtitlenamewidth}{12cm}
\renewcommand*{\namefont}{\fontsize{24}{29}\mdseries\upshape}

\firstname{DOHMATOB}
\familyname{Elvis Dopgima}
\title{Research Engineer}              
\address{Parietal - INRIA, CEA / Neurospin Bât 145}{Point Courrier 156,
91191 Gif/Yvette, France.}
%\mobile{<Tel portable>}                    
\email{elvis.dohmatob.inria.fr}                      
\homepage{www.linkedin.com/pub/elvis-dohmatob/79/ba9/53b}

\begin{document}
\maketitle

\section{Education}
\cventry{2010-2011}{MSc. in Cryptology and Information Security}{University of Bordeaux
1}{}{}{Pentesting for telecom and VoIP-like protocols including SS7, SIGTRAN, SIP, GTP, etc.}

\cventry{2009-2010}{Maîtrise ès Mathématiques}{University of Bordeaux 1}{}{}
{On explicit constructions of ``good'' LDPC QECCs (\emph{Low-Density Parity-Check Quantum Error-Correcting Codes}). Supervised by Gilles ZEMOR}
\cventry{2005-2008}{BSc. in Mathematics and Computer Science}{University of Buea, Cameroon}{}{}
{}

\section{Professional Experience}
\cventry{October 2012 - present}{Research engineer}{PARIETAL Team - INRIA, Neurospin CEA, Saclay}{}{}{
Non-smooth convex optimization; preprocessing and statistical analysis of fMRI data; registration algorithms;
machine learning on fMRI data; software engineering\newline{}}
\cventry{September 2011 - October 2012}{Freelancer and Open-Source}{Various employers}{}{}{
Simulations for CR (Cognitive Radio) research; Windows system programming (DLLs, user-space
root-kits, etc.); implementation of Machine Learning algorithms\newline{}}

\cventry{March 2011 - August 2011}{Cryptology and Security intern}{P1 Security}{Paris}{France}{
Implementation of an event-driven pentesting framework for telecom and VoIP-like protocols
\newline{}}

\section{IT and Computing Skills}
%\footnotesize{\emph{Software I have developed or contributed to can be found at https://github.com/dohmatob}}
\cvitem{Languages}{Python, ASM x86, C/C++, MATLAB, R, PARI/GP, javascript}
\cvitem{Maching Learning}{LibSVM, scikit-learn, pandas}
\cvitem{Neuro-imaging}{nilearn, SPM, FSL, nipy, nipype, freesurfer, mayavi, pypreprocess}
\cvitem{Code Engineering}{OOP, TDD, EDD, version control (git, github), CI (travis), parallel computing}
\cvitem{Operating Systems}{Linux, Windows (including shell scripting and system programming skills)}
\cvitem{Network Protocols}{TCP/IP, SMB, IPSec, LDAP, SSL, SIP, DNS}
\cvitem{Cryptology}{Number Theory, Elliptic Curves, Smart Cards, Asymmetric Cryptography (RSA), Symmetric
Cryptography (PKI, DH, DES, AES)}
\cvitem{Security}{Snort, Nmap, METASPLOIT, OllyDbg, Immunity Debugger, IDA Pro, SPIKE}

\section{Scientific Publications (journal and conference papers)}
\cvitem{PRNI 2014 (\emph{IEEE})}{E. DOHMATOB, A. Gramfort, B. THIRION, G. Varoquaux
``\emph{Benchmarking solvers for least-squares and logistic regression in brain imaging}''.
Pattern Recoginition in Neuroimaging (PRNI), IEEE. \url{http://hal.inria.fr/hal-00991743}}
\cvitem{MICCAI 2013}{ A. Abraham, E. DOHMATOB, B. THIRION, D. SAMARAS, and G. VAROQUAUX,
``\emph{Extracting brain regions from rest fMRI with Total-Variation constrained dictionary learning}''.
MICCAI - 16th International Conference on Medical Image Computing and Computer Assisted Intervention - 2013 (2013).
\url{http://hal.inria.fr/hal-00853242}}

\section{Contributions to open-source software projects}
\cvitem{Neuro-Imaging}{nipy \url{http://nipy.org}, nilearn \url{http://nilearn.github.io},
  pypreprocess \url{https://github.com/neurospin/pypreprocess}}
\cvitem{Personal projects}{See complete list on my github profile: \url{https://github.com/dohmatob}}

\section{Scientific Talks}
\cvitem{PRNI 2014}{At the PRNI (Pattern Recoginition in Neuroimaging) conference that toke place
3rd -- June 6th 2014 (Max-Planck Institute for Intelligent Systems, Tuebingen -- Germany) I presented my work,
\emph{Benchmarking solvers for least-squares and logistic regression in brain imaging}
(\url{http://hal.inria.fr/hal-00991743})%% , under the ``Advances in fMRI analysis'' section of the conference
  .}

\section{Hackathon Experience}
\cvitem{Parietal retreat 2014}{During the last retreat of our team (Parietal -- INRIA) to Normandy (6th -- 8th April 2014),
Virgile FRITSCH and I did VBM (Voxel-Based Morphometry) on a public dataset (Oasis database). The outcome of this
sprint is summarized here \url{https://github.com/Parietal-INRIA/parietal-python/wiki/VBM-dataset-for-nilearn}}
\cvitem{Google Hash Code Paris 2014}{In this competition (4th -- 5th April 2014), I teamed with 2 other members
realize the task of implementing a street-viewer for Paris. The underlying problem can formulated as a multi-objective TSP.
Our algorithm was a Monte-Carlo (random walks on the roadmap of Paris).}
\cvitem{Brainhack 2013}{The hackathon held 23rd -- 26th October 2013 in Paris. With Alexandre Gramfort,
I worked on Henson's multimodal (fMRI, EEG/MEG, DTI) faces vs objects dataset.}

\section{Languages}
\cvline{Bilingual}{English (fluent), French (fluent)}

\section{Scholarships}
\cvitem{2009 - 2011}{Erasmus Mundus, University of Bordeaux 1}

\section{Interests}
\cvline{Research}{Machine learning, optimization, image registration, stochastics and statistics,
cryptology, human connectome mapping}
\cvline{Hobbies}{Reading, dancing, running}

\end{document}
