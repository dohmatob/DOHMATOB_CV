\documentclass[11pt,a4paper,roman,unicode]{moderncv}
\PassOptionsToPackage{
  pdfpagelabels=false,
  urlcolor=magenta,
  colorlinks=True,
  citecolor=gray}{hyperref} 

% fonts
\usepackage{mathpazo}
%% % URls
%% \usepackage{hyperref}
%% \hypersetup{
%%   colorlinks=true,
%%   citecolor=gray,
%%   urlcolor=magenta, %couleur des hyperliens
%% }

\moderncvtheme[blue]{classic}                
\usepackage[utf8]{inputenc}
\usepackage[top=1.1cm, bottom=1.1cm, left=2cm, right=2cm]{geometry}
%\usepackage[scale=0.8]{geometry}

\setlength{\hintscolumnwidth}{3.96cm}
% \setlength{\makecvtitlenamewidth}{12cm}
\setlength{\makecvheadnamewidth}{12cm}
\renewcommand*{\namefont}{\fontsize{24}{29}\mdseries\upshape}

\firstname{Elvis}
\familyname{Dohmatob}
\title{Curriculum Vitae}
\extrainfo{Date of birth: 27 April 1987
  %\\Nationality: Cameroon
}
\title{PhD student}   
\email{elvis.dohmatob.inria.fr}
% \photo[64pt][0.4pt]{elvis}
\address{Neurospin CEA, Bât 145}{Point Courrier 156,
91191 Gif/Yvette, France.}
%\mobile{06-46-85-29-39}
\homepage{dohmatob.github.io}

\begin{document}
\maketitle
\section{Education}
\cventry{Oct 2014 -- present}{PhD Student, Computer Science}{Parietal Team,
  INRIA / CEA, Neurospin, Universit\'e Paris-Saclay, France}{}{}
        {
          \textbf{Graduation date:} September 2017.\\
          \textbf{Supervisors:} Bertrand THIRION and Gael VAROQUAUX\\
          \textbf{Title:} Enhancement of functional brain connectome (connectivity / covariance matrices, etc.) analysis
          by the use of deformable models in the estimation of spatial decompositions of the brain images.
         More details on my blog at \url{http://dohmatob.github.io}.}
\cventry{2010 -- 2011}{MSc. Cryptology and Information Security}{University of Bordeaux
1}{}{}{Pentesting telecom and VoIP-like protocols like SS7, SIGTRAN, SIP, GTP.}
\cventry{2009 -- 2010}{Maîtrise ès Mathématiques}{University of Bordeaux 1}{}{}
{On explicit constructions of ``good'' LDPC QECCs (\emph{Low-Density Parity-Check Quantum Error-Correcting Codes}). Supervised by Gilles ZEMOR.}
\cventry{2005 -- 2008}{BSc. Mathematics and Computer Science}{University of Buea}{}{}
{}

\section{Selected scientific publications}
\textbf{Summary from Google scholar:} Total citations $ \ge 169$; total papers $ \ge 15$; h index $ \ge 3$; $110$ index $ \ge 3$.\\
Full information available at: \url{https://scholar.google.fr/citations?user=FDWgJY8AAAAJ&hl=fr}\\\\
\cvitem{2016}{
  \begin{itemize}
  \item{\emph{Learning brain regions via large-scale online structured sparse dictionary learning}. Advanced Neural Information Processing Systems -- NIPS conference.
  \url{https://hal.inria.fr/hal-01369134v3}}
    \item{\emph{A simple algorithm for computing Nash-equilibria in incomplete information games}. NIPS OPT2016 workshop. \url{https://arxiv.org/abs/1507.07901}}
  \end{itemize}
}
\cvitem{2015}{
  \begin{itemize}
  %\item{E. Dohmatob
    \item{\emph{Local Q-Linear Convergence and Finite-time Active Set Identification of ADMM on a Class of Penalized Regression Problems}}. ICASSP - 41st International Conference on Acoustics, Speech and Signal Processing (\emph{IEEE}). \url{https://hal.archives-ouvertes.fr/hal-01265372/file/paper.pdf}
  \item{\emph{Integrating Multi-modal Priors in Predictive Models for the Functional Characterization of Alzheimer’s Disease}. MICCAI -- 18th International Conference on Medical Image Computing and Computer Assisted Intervention. \url{https://hal.archives-ouvertes.fr/hal-01174636/file/paper983.pdf}
}
  \end{itemize}
}
  
\cvitem{2014}{\begin{itemize}
  \item{\emph{Region segmentation for sparse decompositions: better brain parcellations from rest fMRI}.
\url{http://stmi2014.ece.cornell.edu/papers/STMI-P-9.pdf}}
  \item{\emph{Which fMRI clustering gives good brain parcellations?}.
Frontiers in Neuroinformatics. \url{http://journal.frontiersin.org/Journal/10.3389/fnins.2014.00167/abstract}}
  \item{\emph{Benchmarking solvers for TV-$\ell_{1}$  least-squares and logistic regression in brain imaging}.
    PRNI - Pattern Recognition in Neuro-Imaging (\emph{IEEE}). \url{http://hal.inria.fr/hal-00991743}}
\end{itemize}}
\cvitem{2013}{
\begin{itemize}
\item{\emph{Extracting brain regions from rest fMRI with Total-Variation constrained dictionary learning}.
  MICCAI - 16th International Conference on Medical Image Computing and Computer Assisted Intervention. \url{http://hal.inria.fr/hal-00853242}}
\end{itemize}
}

\section{Scientific reviewing}
\cvitem{2016}{NIPS --Advanced Neural Information Processing Systems}

\section{Selected workshops \& and Symposia}
\cvitem{2017}{\begin{itemize}
  \item Attended two-week-long machine-learning summer school (MLSS) in Tuebingen, Germany.
  \end{itemize}
}

\cvitem{2016}{\begin{itemize}
  \item Taught at Nilearn (machine learning in neuroimaging) workshop at BrainHack,
    Lausanne, Switzerland, in June.
    \item Taught at Nilearn workshop at OHBM, Geneva, Switzerland, in June.
  \item Taught at workshop on Python programming and machine learning, at Psychiatry department, RWTH, Aachen, Germany, in January.
%%       \item Poster presentation on\textit{``Inter-subject highres EPI-to-EPI direct nonlinear registration outperforms classical T1-based method
%% ``},  OHBM, Geneva, Switzerland.
  \end{itemize}
}

%% \cvitem{2015}{\begin{itemize}
%%     \item Oral + poster presentation on\textit{``SpaceNet:
%% Multivariate brain decoding and segmentation''},  OHBM, Honolulu, Hawaii, USA
%% \item Oral presentation on\textit{``Speeding-up
%% model selection in GraphNet via early-stopping and
%% feature-screening''}, Stanford, USA
%%   \end{itemize}
%%   }

%% \cvitem{2014}{\begin{itemize}
%%     \item At the PRNI --Pattern Recognition in Neuro-Imaging-- IEEE  conference that took place
%% 3rd -- 6th June 2014 (Max-Planck Institute for Intelligent Systems, Tuebingen -- Germany), I presented my work,
%% ``\emph{Benchmarking solvers for TV-$\ell_{1}$ least-squares and logistic regression in brain imaging}''
%%   \end{itemize}
%%   }


\section{Professional experience}
\cventry{\llap{Oct 2014 -- present}}{Part-time research engineer}{Parietal Team -- INRIA / CEA, Neurospin, Neurospin, Universit\'e Paris-Saclay, France}{}{}{While preparing my PhD, a 6th of my time is spent programming and consulting.}
\cventry{\llap{Oct 2012 -- Oct 2014}}{Research engineer}{Parietal Team -- INRIA / CEA, Neurospin, Neurospin, Universit\'e Paris-Saclay, France}{}{}{
  software engineering; implementation of structured priors for brain data; optimization;
  preprocessing and statistical analysis of fMRI data; registration algorithms;
  machine learning on fMRI data. Some of the output of this project were contributions to the open-source projects
  \url{https://github.com/neurospin/pypreprocess} and \url{http://nilearn.github.io}.\newline{}}
\cventry{Sep 2011 -- Oct 2012}{Freelancer and Open-Source}{Various employers}{}{}{
Simulations for CR (Cognitive Radio) research; Windows system programming (DLLs, user-space
root-kits, etc.); implementation of Machine Learning algorithms\newline{}}

\cventry{Mar 2011 -- Aug 2011}{Cryptology and Security intern}{P1 Security}{Paris}{France}{
Implementation of an event-driven pentesting framework for telecom protocols
\newline{}}

\section{Languages}
\cvline{Bilingual}{English (fluent), French (fluent)}

\section{Contributions to open-source software projects}
\cvitem{Data science \& AI}{scikit-learn \url{http://scikit-learn.org/stable/}}
\cvitem{Neuro-Imaging}{nilearn \url{http://nilearn.github.io}, nipy \url{http://nipy.org}, 
  pypreprocess \url{https://github.com/neurospin/pypreprocess}}
\cvitem{Complete list}{See complete list on my github profile at \url{https://github.com/dohmatob}}
%% \cvitem{My \emph{Open Source Report Card}}{Tentatively, an impartial automatically generated statistical summary of
%% my ``contributions heat map'' can be found at \url{http://osrc.dfm.io/dohmatob/}}

\section{IT and computing skills}
%\footnotesize{\emph{Software I have developed or contributed to can be found at https://github.com/dohmatob}}
\cvitem{See my github profile at}{\url{https://github.com/dohmatob}}
\cvitem{\llap{Programming Languages}}{Python (including Numpy/Scipy, Maplotlib, Seaborn), bash,
  Latex, C++, Emacs, Matlab}
\cvitem{Data science \& AI}{solid mastery of convex optimization, scikit-learn, pandas, keras}
\cvitem{Neuro-imaging}{nilearn, SPM, FSL, ANTS, nipype, Mango}
\cvitem{Software Engineering}{OOP, TDD, version control (git, github), continuous integration (travis, circle-ci), parallel computing (xargs, joblib)}
\cvitem{Operating Systems}{GNU/Linux, Windows}
%\cvitem{Ethical hacking}{nmap, Metasploit, ImmunityDebugger, Sully}
%% \cvitem{Network Protocols}{TCP/IP, SMB, IPSec, LDAP, SSL, SIP, DNS}
%% \cvitem{Cryptology}{Number Theory, Elliptic Curves, Smart Cards, Asymmetric Cryptography (RSA), Symmetric
%% Cryptography (PKI, DH, DES, AES)}
%% \cvitem{Security tools}{Snort, Wireshark, Nmap, METASPLOIT, OllyDbg, Immunity Debugger, IDA Pro, SPIKE}


\section{Business experience}
\cvitem{\llap{2016}}{Participated in ``Doctoriales 2016 projet innovant'' in which I collaborated with a team of 7 other participants to build a start-up in 24 hours.}

\section{Hackathon experience}
%% \cvitem{\llap{Parietal retreat,}\\ 6th -- 8th April 2014}{Virgile FRITSCH and I did VBM (Voxel-Based Morphometry) on a public dataset (Oasis database). The outcome of this
%% sprint is summarized here \url{https://github.com/Parietal-INRIA/parietal-python/wiki/VBM-dataset-for-nilearn}}
\cvitem{\llap{2013 -- present}}{BrainHack Lausanne (2016); BrainHack Paris (2016); scikit-learn coding sprint Paris (2015); PyData Paris (2015);
  Google Hash Code Paris (2014); BrainHack Paris (2013)}


\section{Awards and scholarships}
\cvitem{2014}{Honourable Mention (2ND price) awarded to the paper ``\emph{Benchmarking solvers for
TV-$\ell_{1}$ least-squares and logistic regression in brain imaging}'' (\url{http://hal.inria.fr/hal-00991743}), presented at the 4th international
workshop on Pattern Recognition in Neuro-imaging (PRNI 2014), Max-Planck Institute for Intelligent Systems,
Tuebingen -- Germany}
\cvitem{2009 - 2011}{Erasmus Mundus, ALGANT (\textit{Algebra, Geometry, and Number Theory}), Université de Bordeaux 1}

\section{Interests}
\cvline{Research}{data science \& AI, convex optimization, neuroscience, game theory}
\cvline{Hobbies}{programming, dancing, ping-pong, arcade games}

\end{document}
